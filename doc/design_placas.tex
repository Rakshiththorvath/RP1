\chapter{Desenho e Prototipagem das Placas Mãe, Placas de Chute e Placas Acionadoras dos Motores dos Robôs}\label{cap:design_placas}

O projeto das placas foi feito em duas etapas. A primeira etapa foi o desenvolvimento do esquemático do circuito, representação simbólica dos componentes e suas ligações. Nessa fase foram selecionados os componentes. A segunda etapa foi o layout, em que foi feito um CAD (Computer Aided Design ­– desenho assistido por computador) com o posicionamento dos componentes e o desenho das trilhas de cobre, tomando por base o esquemático. Foi preciso levar em conta a corrente que circularia pelo circuito para definir a larguras das trilhas, os tamanhos dos componentes a serem soldados e o tamanho da placa que seria produzida. Ao projetar as placas, os alunos da iniciativa adquirem experiência tanto em desenho de PCBs (Printed Circuit Board – placa de circuito impresso), como em soldagem de componentes SMD (Surface Mounted Device – dispositivo montado em superfície).

Para o desenvolvimento das placas, foi utilizado o software Altium Designer. As FIGURAS ?????? apresentam os diagramas de blocos feitos a partir de esboços dos desenhos das três placas, utilizando uma das funcionalidades do Altium.

Na placa mãe, o módulo de transmissão envia os dados recebidos da inteligência para o microcontrolador (que fica no módulo de controle adquirido) via protocolo SPI.
O módulo de controle envia sinais PWM para os módulos de motores e um pulso e um sinal digital para o módulo do chute. As correntes nos módulos de motores são medidas por circuitos integrados dedicados e enviadas para o módulo de controle via protocolo I2C.

Para funcionamento dos motores, são usados drivers e uma ponte H, para que os motores possam trabalhar nos dois sentidos.
Feito dessa forma, o hardware passa a ter o fluxograma apresentado na FIGURA. Os comandos chegam através do módulo de comunicação e pela placa mãe é transmitida até a o módulo de controle (Discovery STM32F4), que então a envia aos módulos pertinentes, motores ou chute. Há quatro motores ligados às rodas e um ao drible. O chute está separado em forte (baixo) e fraco (alto). Para fazer o controle das velocidades de giro de cada roda, informações sobre as rotações das rodas são passadas para a Discovery, que pode também passá-las para o módulo de transmissão, se for o caso.


\input{doc/placa_mae}
\input{doc/modulo_motor}
\section{Módulo do Chute}\label{sec:modulo_chute}

O mecanismo de chute do robô foi elaborado utilizando dois solenóides como atuadores, um responsável pelo chute para frente e outro responsável pelo chute alto. E a placa (FIGURA) é responsável por produzir a alta tensão necessária para ativar os dois solenóides, e transmitir a energia acumulada para acionar os atuadores.


Figura 9 - Módulo do chute
A placa funciona através de um circuito de conversão DC-DC controlado pelo circuito integrado MC34063 que converte os 7,8V DC da bateria em uma saída de 180V DC ligada a dois capacitores eletrolíticos de 2200 $\mu$F e 200V. Além desse circuito existe ainda um circuito para chavear a tensão dos capacitores nos solenóides, o que é feito utilizando um TC4427 driver de MOSFET e dois MOSFETs de potência IRFP4868PBF. 
O controle desse módulo é feito através da classe chute que utiliza a classe degrau unitário para limitar a energia transmitida a bola alterando a duração do intervalo de tempo que os MOSFETs ficarão ativados.

Figura 10 - Diagrama de blocos do módulo de chute



% vim: tw=80 et ts=2 sw=2 sts=2 ft=tex spelllang=pt_br,en


% Ramificação constante ou taxa constante

% vim: tw=80 et ts=2 sw=2 sts=2 ft=tex spelllang=pt_br,en
